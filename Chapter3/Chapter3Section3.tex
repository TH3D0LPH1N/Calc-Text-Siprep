\textbf{\underline{\large{3.3: Context For Definite Integrals: Area, Displacement, and Net Change}}} \par

Since we originally defined the definite integral in terms of ``area under a curve,'' we need to consider what this idea of ``area'' really means in relation to the definite integral. \par

Recall \textbf{Ex 3.2.2}, where we had a function $y = x\cos \left(x^2\right)$ on $x \in \left[0 \, \sqrt{\pi}\right]$. The graph looks like this: \\

\begin{center}
    \includegraphics*[width = 0.6\textwidth]{\graphicsdir Chapter 3 Graphics/3.3-Graphic1.png}
\end{center}

In \textbf{Ex 3.2.2}, we found that $\int_0^{\sqrt{\pi}} x\cos \left(x^2\right) \, dx = 0 \forcespace$. But, there's clearly area under the curve, so how can the integral equal both the area and 0? Well, as it turns out, because the integral was created from rectangles with width $dx$ and height $f(x)$, a negative $f(x)$ will result in a rectangle with ``negative area.'' Take a look at the following graph: \\

\begin{center}
    \includegraphics*[width = 0.6\textwidth]{\graphicsdir Chapter 3 Graphics/3.3-Graphic2.png}
\end{center}

It's clear that by our definition of the definite integral, the ``area'' in red would cancel out the ``area'' in blue. So, how would we find the actual, positive area? That is what we are going to talk about in this section. \par

\begin{tcolorbox}[objective]
    \begin{center}
        OBJECTIVES \\[11pt]
    \end{center}
    Relate Definite Integrals to Area Under a Curve. \\
    Understand the Difference Between Displacement and Distance. \\
    Understand Displacement and Distance in Other Contexts.
\end{tcolorbox} \vspace{11pt}

\begin{tcolorbox}[example]
    \textbf{Ex 3.3.1: } What is the area under $y = x\cos \left(x^2\right)$ on $x \in \left[0, \sqrt{x}\right]$?
\end{tcolorbox}
\begin{tcolorbox}[solution]
    \textbf{Sol 3.3.1: } First, let's make sure we're clear on terminology. In this context, ``area under'' means ``area between the graph and $x$-axis.'' Now, to find the area under this graph, we have two methods. \par

    The first method is 
\end{tcolorbox}