\textbf{\underline{\large{3.1: The Fundamental Theorem of Calculus}}} \par

Up to this point, we've seen how integration can be used to find the exact area under a curve by taking the limit of Riemann sums. But what's truly remarkable is how integration connects so deeply with differentiation. This connection is captured in one of the most important results in all of calculus: the Fundamental Theorem of Calculus (FTC). \par

\begin{center}
    \fbox{\fbox{\begin{minipage}{0.96\textwidth}
        \vspace{11pt}
        \begin{center}
            \textbf{The Fundamental Theorem of Calculus}
        \end{center}
        \vspace{11pt}
        If $f(x)$ is a continuous function on $[a, b]$, then: \par
        \vspace{11pt}
        \begin{enumerate}
            \item $\diff \int_c^x f(t) \, dt = f(x)$ or $\diff \int_c^u f(t) \, dt = f(u) \cdot \dfrac{du}{dx}$ \\[11pt]
            \item If $F'(x) = f(x)$, then $\int_a^b f(x) \, dx = F(b) - F(a)$.
        \end{enumerate}
        \vspace{11pt}
    \end{minipage}}}
\end{center}

The first part of the Fundamental Theorem of Calculus simply says what we already know--that an integral is an anti-derivative. The second part of the Fundamental Theorem says that the answer to a definite integral is the difference between the anti-derivative at the upper bound and the anti-derivative at the lower bound. \par

The idea of the integral meaning the area may not make sense initially, mainly because we are used to geometry, where an area is always measured in square units. But, that is only because the length and width are always measured in the same kind of units, so multiplying length and width area always measured in the same kind of units, so multiplying length and width must yield square units. We are expanding our vision beyond that narrow view of things here. Consider a graph where the $x$-axis is time and the $y$-axis is velocity in feet per second. The area under the curve would be measured as seconds multiplied by feet per seconds, which is simply feet. So, the area under the curve is equal to the distance traveled in feet. In other words, the integral of velocity is distance. \par

\begin{tcolorbox}[objective]
    \begin{center}
        OBJECTIVES \\[11pt]
    \end{center}
    Evaluate Definite Integrals. \\
    Find Average Value of a Continuous Function Over a Given Interval. \\
    Differentiate Integral Expressions with the Variable in the Boundary.
\end{tcolorbox}

Let us first consider part 2 of the Fundamental Theorem of Calculus, since it has a very practical application. This part of the Fundamental Theorem of Calculus gives us a clear method for evaluating definite integrals. \par

\begin{tcolorbox}[example]
    \textbf{Ex 3.1.1: } Evaluate $\int_2^8 (4x + 3) \, dx$
\end{tcolorbox}
\begin{tcolorbox}[solution]
    \textbf{Sol 3.1.1: } First, let's start by treating this as a regular antiderivative \begin{align*}
        \int (4x + 3) \, dx = 2x^2 + 3x
    \end{align*}
    Note that our $+C$ will not be needed, as we will be taking a definite integral. Now, let's apply the Fundamental Theorem of Calculus. \begin{align*}
        \int_2^8 f(x) \, dx &= F(8) - F(2) \\[11pt]
        & = \left(2x^2 + 3x\right) \eval_2^8 \\[11pt]
        & = 2(8)^2 + 3(8) - \left(2(2)^2 + 3(2)\right) \\[11pt]
        & = \boxed{138}
    \end{align*}
    The vertical bar that you see is called the evaluation bar, and it's used to indicate that we are evaluating the antiderivative at the upper and lower limits of integration.
\end{tcolorbox} \vspace{11pt}

\begin{tcolorbox}[example]
    \textbf{Ex 3.1.2: } Evaluate $\int_1^4 \dfrac{1}{\sqrt{x}} \, dx$ 
\end{tcolorbox}
\begin{tcolorbox}[solution]
    \textbf{Sol 3.1.2: } \begin{align*}
        \int_1^4 \dfrac{1}{\sqrt{x}} \, dx &= 2\sqrt{x} \eval_1^4 \\[11pt]
        & = 2\sqrt{(4)} - 2\sqrt{(1)} \\[11pt]
        & = \boxed{2}
    \end{align*}
\end{tcolorbox} \vspace{11pt}

\begin{tcolorbox}[example]
    \textbf{Ex 3.1.3: } Evaluate $\int_0^{\frac{\pi}{2}} \sin (x) \, dx$
\end{tcolorbox} 
\begin{tcolorbox}[solution]
    \textbf{Sol 3.1.3: } \begin{align*}
        \int_0^{\frac{\pi}{2}} \sin (x) \, dx &= -\cos (x) \eval_0^{\frac{\pi}{2}} \\[11pt]
        & = -\cos \left(\dfrac{\pi}{2}\right) + \cos(0) \\[11pt]
        & = \boxed{1}
    \end{align*}
\end{tcolorbox} \vspace{11pt}

\begin{tcolorbox}[example]
    \textbf{Ex 3.1.4: } Evaluate $\int_1^2 \dfrac{4 + u^2}{u^3} \, du$
\end{tcolorbox}
\begin{tcolorbox}[solution]
    \textbf{Sol 3.1.4: } \begin{align*}
        \int_1^2 \dfrac{4 + u^2}{u^3} \, du &= \int_1^2 \left(4u^{-3} + u^{-1}\right) \, du \\[11pt]
        & = \left(-2u^{-2} + \ln |u|\right) \eval_1^2 \\[11pt]
        & = -2(2)^{-2} + \ln 2 - (-2(1)^{-2} + \ln 1) \\[11pt]
        & = \boxed{\dfrac{3}{2} + \ln 2}
    \end{align*}
\end{tcolorbox} \vspace{11pt}

\begin{tcolorbox}[example]
    \textbf{Ex 3.1.5: } Evaluate $\int_{-5}^5 \dfrac{1}{x^3} \, dx$
\end{tcolorbox}
\begin{tcolorbox}[solution]
    \textbf{Sol 3.1.5: } When initially looking at the problem, one may simply proceed with finding the definite integral. \begin{align*}
        \int_{-5}^5 \dfrac{1}{x^3} \, dx & = -\dfrac{1}{x^2} \eval_{-5}^5 \\[11pt]
        & = -\dfrac{1}{(-5)^2} + \dfrac{1}{(5)^2} \\[11pt]
        & = 0
    \end{align*}
    But, this is a trap! We have to be careful here, because the function $\dfrac{1}{x^3}$ is \textit{not defined} over the the interval $[-5, 5]$, since $\dfrac{1}{0^3}$ is undefined. Therefore, the Fundamental Theorem of Calculus $\boxed{\text{does not apply}}$.
\end{tcolorbox}

Just as we had many properties for the indefinite integral, we have many properties for the definite integral. There are three main properties that are utilized often on the AP exam. \par

\begin{center}
    \fbox{\fbox{\begin{minipage}{0.96\textwidth}
        \vspace{11pt}
        \begin{center}
            \textbf{Properties of Definite Integrals}
        \end{center}
        \vspace{11pt}
        \begin{enumerate}
            \item $\int_a^b f(x) \, dx = -\int_b^a f(x) \, dx$ \\[11pt]
            \item $\int_a^a f(x) \, dx = 0$ \\[11pt]
            \item $\int_a^b f(x) \, dx = \int_a^c f(x) \, dx + \int_c^b f(x) \, dx$, where $a < c < b$
        \end{enumerate}
        \vspace{11pt}
    \end{minipage}}}
\end{center}

\begin{tcolorbox}[example]
    \textbf{Ex 3.1.6: } If $\int_{-5}^2 f(x) \, dx = -17$ and $\int_5^2 f(x) \, dx = -4$, find $\int_{-5}^5 f(x) \, dx$.
\end{tcolorbox}
\begin{tcolorbox}[solution]
    \textbf{Sol 3.1.6: } \begin{align*}
        \int_{-5}^5 f(x) \, dx &= \int_{-5}^2 f(x) \, dx + \int_2^5 f(x) \, dx \\[11pt]
        & = \int_{-5}^2 f(x) \, dx - \int_5^2 f(x) \, dx \\[11pt]
        & = -17 - (-4) \\[11pt]
        & = \boxed{-13}
    \end{align*}
\end{tcolorbox}

Part I of the Fundamental Theorem of Calculus is very important for the $\textbf{theory}$ of calculus, but is limited (hehe) in the context of this course to L'Hospital problems which will will explore in a later chapter. Here is how the formula may be applied. \par

\begin{tcolorbox}[example]
    \textbf{Ex 3.1.7: } Use the Fundamental Theorem of Calculus to find $f'(t)$ if $f(t) = \int_2^{3t^2} (4x + 3) \, dx$ 
\end{tcolorbox} 
\begin{tcolorbox}[solution]
    \textbf{Sol 3.1.7: } \begin{align*}
        f'(t) &= \dfrac{d}{dt} \int_2^{3t^2} (4x + 3) \, dx \\[11pt]
        & = \left(4\left(3t^2\right) + 3\right)(6t) \\[11pt]
        & = \boxed{72t^3 + 18t}
    \end{align*}
\end{tcolorbox}

\newpage

\textbf{\large{3.1 Free Response Homework}} \par

Use part II of the Fundamental Theorem of Calculus to evaluate the integral, or explain why the integral cannot be evaluated. \par

\twoquestion{1. $\int_{-1}^3 x^5 \, dx$}{2. $\int_2^7 (5x - 1) \, dx$} \\[11pt]
\twoquestion{3. $\int_{-5}^5 \dfrac{2}{x^3} \, dx$}{4. $\int_{-3}^{-1} \dfrac{x^7 - 4x^3 - 3}{x} \, dx$} \\[11pt]
\twoquestion{5. $\int_1^2 \dfrac{3}{t^4} \, dt$}{6. $\int_{\frac{\pi}{4}}^{\frac{3\pi}{4}} \csc (y)\cot (y) \, dy$} \\[11pt]
\twoquestion{7. $\int_0^{\frac{\pi}{4}} \sec^2 (y) \, dy$}{8. $\int_1^9 \dfrac{3}{2z} \, dz$} \\[11pt]
\twoquestion{9. $\int_1^8 \dfrac{x^2 - 4}{\sqrt[3]{x}} \, dx$}{10. $\int_{\pi}^{\frac{5\pi}{4}}\sin (y) \, dy$} \\[11pt]
\twoquestion{11. $\int_1^4 \dfrac{x^4 - 4x^2 - 5}{x^2} \, dx$}{12. $\int_3^5 \left(x^2 + 5x + 6\right) \, dx$} \\[11pt]
\twoquestion{13. $\int_{\pi}^{\frac{\pi}{4}} \cos (y) \, dy$}{14. $\int_1^4 \dfrac{x^3 - 2x^2 - 4x}{x^2} \, dx$} \\[11pt]
\twoquestion{15. $\int_1^2 \dfrac{x^2 - 4x + 7}{x} \, dx$}{16. $\int_1^{16} \dfrac{2x^2 - 1}{\sqrt[4]{x}}$} \\[11pt]

Use the following values for problems 17 - 27 to evaluate the given integrals. \begin{align*}
    \arraycolsep=55pt\def\arraystretch{2.2} 
    \begin{array}[c]{|c|c|}
        \hline
        \int_{-2}^5 f(x) \, dx = -2 & \int_1^5 f(x) \, dx = 3 \\[5.5pt] \hline
        \int_{-2}^1 g(x) \, dx = 4 & \int_5^1 g(x) \, dx = 9 \\[5.5pt] \hline
        \int_1^5 h(x) \, dx = 7 & \int_5^{-2} h(x) \, dx = -6 \\[5.5pt] 
        \hline
    \end{array}
\end{align*}

\twoquestion{17. $\int_{-2}^1 f(x) \, dx$}{18. $\int_{-2}^5 g(x) \, dx$} \\[11pt]
\twoquestion{19. $\int_{-2}^1 h(x) \, dx$}{20. $\int_1^5 (f(x) - g(x)) \, dx$} \\[11pt]
\twoquestion{21. $\int_{-2}^5 (g(x) + h(x)) \, dx$}{22. $\int_{-2}^1 (h(x) - f(x)) \, dx$} \\[11pt]
\twoquestion{23. $\int_{-2}^5 (h(x) + f(x)) \, dx$}{24. $\int_1^5 (2f(x) + 3h(x)) \, dx$} \\[11pt]
\twoquestion{25. $\int_{-2}^1 (2f(x) - 3g(x)) \, dx$}{26. $\int_{-2}^5 \left(\dfrac{1}{2}g(x) + 4h(x)\right) \, dx$} \\[11pt]
\twoquestion{27. $\int_1^5 \left(\dfrac{1}{3}h(x) + 2f(x)\right) \, dx$}{28. $\int_5^5 (f(x) + g(x) + h(x)) \, dx$} \\[11pt]

Use part I of the Fundamental Theorem of Calculus to find the derivative of the function. \par

\twoquestion{29. $g(y) = \int_2^y t^2\sin (t) \, dt$}{30. $g(x) = \int_0^x \sqrt{1 + 2t} \, dt$} \\[11pt]
\twoquestion{31. $F(x) = \int_x^2 \cos \left(t^2\right) \, dt$}{32. $h(x) = \int_2^{\frac{1}{x}} \arctan (t) \, dt$} \\[11pt]
\twoquestion{33. $y = \int_3^{\sqrt{x}} \dfrac{\cos (t)}{t} \, dt$}{34. $f(x) = \int_e^{x^2} \ln \left(t^2 + 1\right) \, dt$} \\[11pt]
\twoquestion{35. $f(x) = \int_{10}^{x^2} t\ln t \, dt$}{36. $f(x) = \int_{e^x}^5 \left(t^3 + t + 1\right) \, dt$} \\[11pt]
\twoquestion{37. If $F(x) = \int_1^{t^2} \dfrac{\sqrt{1 + u^4}}{u} \, du$, find $F'(t)$.}{38. If $h(x) = \int_{\pi}^{\sqrt{x}} e^{5t} \, dt$, find $h'(x)$.} \\[11pt]
\twoquestion{39. If $h(m) = \int_5^{\cos (m)} t^2\cos^{-1} (t) \, dt$, find $h'(m)$.}{40. If $h(y) = \int_5^{\ln y} \dfrac{e^t}{t^4} \, dt$, find $h'(y)$.} \\[11pt]

\textbf{\large{3.1 Multiple Choice Homework}} \par

\begin{questions}
    \question If $\int_{-5}^2 f(x) \, dx = -17$ and $\int_5^2 f(x) \, dx = -4$, then $\int_{-5}^5 f(x) \, dx = $ \\

    \begin{oneparchoices}
        \choice $-21$
        \choice $-13$
        \choice $0$
        \choice $13$
        \choice $21$
    \end{oneparchoices} \par \horizontalline

    \question Let $f$ and $g$ be continuous functions such that $\int_0^6 f(x) \, dx = 9$, $\int_3^6 f(x) \, dx = 5$, and $\int_3^0 g(x) \, dx = -7$. What is the value of $\int_0^3 \left(\dfrac{1}{2}f(x) - 3g(x)\right) \, dx$. \\

    \begin{oneparchoices}
        \choice $-23$
        \choice $-19$
        \choice $-\dfrac{17}{2}$
        \choice $19$
        \choice $23$
    \end{oneparchoices} \par \horizontalline

    \question Given that $\int_2^3 P(t) \, dt = 7$ and $\int_2^7 P(t) \, dt = -2$, what is $\int_7^3 P(t) \, dt$? \\

    \begin{oneparchoices}
        \choice $-9$
        \choice $-5$
        \choice $5$
        \choice $9$
        \choice not enough information
    \end{oneparchoices} \par \horizontalline

    \question Based on the information below, find $\int_1^{-2} (g(x) + f(x)) \, dx$ \begin{align*}
        \arraycolsep=55pt\def\arraystretch{2.2} 
        \begin{array}[c]{|c|c|}
            \hline
            \int_{-2}^5 f(x) \, dx = -2 & \int_1^5 f(x) \, dx = 3 \\[5.5pt] \hline
            \int_{-2}^1 g(x) \, dx = 4 & \int_5^1 g(x) \, dx = 9 \\[5.5pt] 
            \hline
        \end{array}
    \end{align*}

    \begin{oneparchoices}
        \choice $-9$
        \choice $-1$
        \choice $0$
        \choice $1$
        \choice $9$
    \end{oneparchoices} \par \horizontalline

    \question Based on the information below, find $\int_5^{-2} (g(x) - f(x)) \, dx$ \begin{align*}
        \arraycolsep=55pt\def\arraystretch{2.2} 
        \begin{array}[c]{|c|c|}
            \hline
            \int_{-2}^5 f(x) \, dx = -2 & \int_1^5 f(x) \, dx = 3 \\[5.5pt] \hline
            \int_{-2}^1 g(x) \, dx = 4 & \int_5^1 g(x) \, dx = 9 \\[5.5pt] 
            \hline
        \end{array}
    \end{align*}

    \begin{oneparchoices}
        \choice $-3$
        \choice $3$
        \choice $6$
        \choice $-6$
        \choice $14$
    \end{oneparchoices} \par \horizontalline

    \question Using the table values from questions 5 and 6, which of the following cannot be determined? \\

    \begin{oneparchoices}
        \choice $\int_5^1 (g(x) + f(x)) \, dx$
        \makebox[0.30\textwidth] \choice $\int_1^{-2} (g(x) - f(x)) \, dx$ \\[11pt]
        \makebox[0.035\textwidth] \choice $\int_{-2}^5 3g(x)(-4(f(x))) \, dx$ 
        \makebox[0.26\textwidth] \choice $\int_1^5 (3g(x) + 4f(x)) \, dx$ 
    \end{oneparchoices} \par \horizontalline
\end{questions}