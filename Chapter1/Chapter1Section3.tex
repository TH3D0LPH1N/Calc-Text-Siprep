\textbf{\underline{\large{1.3: Trig, Trig Inverse, and Log Rules}}} \par

Remember:
\begin{center}
    \fbox{\fbox{\begin{minipage}{0.96\textwidth}
        \vspace{11pt}
        \begin{align*}
            & \text{The Product Rule: } f'(x) = U \cdot \dfrac{dV}{dx} + V \cdot \dfrac{dU}{dx} \\[11pt]
            & \text{The Quotient Rule: } f'(x) = \dfrac{V \cdot \dfrac{dU}{dx} - U \cdot \dfrac{dV}{dx}}{V^2} \\[11pt]
        \end{align*}
    \end{minipage}}}
\end{center}

\begin{center}
    \fbox{\begin{minipage}{0.96 \textwidth}
        \begin{center}
            OBJECTIVES \\[11pt]
        \end{center}
        Find the Derivative of a Product or Quotient of Two Functions. 
    \end{minipage}}
\end{center}

\newpage

\hypertarget{Product Rule}{\textbf{\large{The Product Rule}}} \par

\textbf{Ex 1.3.1: } $\diff \left[x^2\sin (x)\right]$ \\[11pt]
\textbf{Sol 1.3.1: } \begin{align*}
    \diff \left[x^2\sin (x)\right] & = x^2 \cdot \cos (x) + \sin (x) \cdot (2x) \\[11pt]
    & = \boxed{x^2\cos (x) + 2(x)\sin (x)}
\end{align*} 

\textbf{Ex 1.3.2: } $\diff \left[5^x \cos (x)\right]$ \\[11pt]
\textbf{Sol 1.3.2: } \begin{align*}
    \diff \left[5^x \cos (x)\right] & = 5^x \cdot (-\sin (x)) + \cos (x) \cdot (5^x \ln 5) \\[11pt]
    & = \boxed{5^x \left(\ln (5)\cos (x) + \sin(x)\right)}
\end{align*}

The product rule is pretty straightforward. The tricky part is simplifying the algebra. \par

\textbf{Ex 1.3.3: } If $f(x) = x^2e^{-\frac{x}{2}}$, find $f'(x)$ \\[11pt]
\textbf{Sol 1.3.3: } \begin{align*}
    & U = x^2, \; \dfrac{dU}{dx} = 2x \\[11pt]
    & V = e^{-\frac{x}{2}}, \; \dfrac{dV}{dx} = e^{-\frac{x}{2}} \cdot \left(-\dfrac{1}{2}\right) = -\dfrac{1}{2}e^{-\frac{x}{2}} \\[11pt]
    & f'(x) \begin{aligned}[t]
        & = x^2 \cdot \left(-\dfrac{1}{2}e^{-\frac{x}{2}}\right) + e^{-\frac{x}{2}} \cdot 2x \\[11pt]
        & = \boxed{xe^{-\frac{x}{2}}\left(-\dfrac{1}{2}x + 2\right)}
    \end{aligned}
\end{align*} 

\textbf{Ex 1.3.4: } $\diff \left[x\sqrt{1 - x^2}\right]$ \\[11pt]
\textbf{Sol 1.3.4: } \begin{align*}
    & U = x, \; \dfrac{dU}{dx} = 1 \\[11pt]
    & V = \sqrt{1 - x^2} = \left(1 - x^2\right)^\frac{1}{2}, \; \dfrac{dV}{dx} = \dfrac{1}{2}\left(1 - x^2\right)^{-\frac{1}{2}} \cdot (-2x) = -\dfrac{x}{\sqrt{1 - x^2}} \\[11pt]
    & \diff \left[x\sqrt{1 - x^2}\right] \begin{aligned}[t]
        & = x \cdot \left(-\dfrac{x}{\sqrt{1 - x^2}}\right) + \sqrt{1 - x^2} \cdot 1 \\[11pt]
        & = \dfrac{-x^2 + (1 - x^2)}{\sqrt{1 - x^2}} \\[11pt]
        & = \boxed{\dfrac{1 - 2x^2}{\sqrt{1 - x^2}}}
    \end{aligned}
\end{align*}

\textbf{Ex 1.3.5: } $\diff \left[(2x - 3)^8\left(3x^2 - 1\right)^7\right]$ \\[11pt]
\textbf{Sol 1.3.5: } \begin{align*}
    & U = (2x - 3)^8, \; \dfrac{dU}{dx} = 8(2x - 3)^7 \cdot 2 = 16(2x - 3)^7 \\[11pt]
    & V = \left(3x^2 - 1\right)^7, \; \dfrac{dV}{dx} = 7\left(3x^2 - 1\right)^6 \cdot 6x  = 42x\left(3x^2 - 1\right)^6\\[11pt]
    & \diff \left[(2x - 3)^8\left(3x^2 - 1\right)^7\right] = (2x - 3)^8 \cdot 42x\left(3x^2 - 1\right)^6 + \left(3x^2 - 1\right)^7 \cdot 16(2x - 3)^7 \\[11pt]
    & \text{This, then, is factorable.} \\[11pt]
    & \diff \left[(2x - 3)^8\left(3x^2 - 1\right)^7\right] \begin{aligned}[t]
        & = 42x(2x - 3)^8\left(3x^2 - 1\right)^6 + 16\left(3x^2 - 1\right)^7 16(2x - 3)^7 \\[11pt]
        & = 2(2x - 3)^7\left(3x^2 - 1\right)^6\left(21x(2x - 3) + 8\left(3x^2 - 1\right)\right) \\[11pt]
        & = 2(2x - 3)^7\left(3x^2 - 1\right)^6\left(42x^2 - 63x + 24x^2 - 8\right) \\[11pt]
        & = \boxed{2(2x - 3)^7\left(3x^2 - 1\right)^6\left(66x^2 - 63x - 8)\right)}
    \end{aligned}
\end{align*}

Remember that in \hyperlink{Section 1.1}{Section 1.1} we said that we would need the Product Rule to deal with the derivative of a function where the variable is in both the base and the exponent. We can now address that situation. \par

\textbf{Ex 1.3.6: } $\diff \left[(\cos (x))^{x^2}\right]$ \\[11pt]
\textbf{Sol 1.3.6: } \begin{align*}
    \diff \left[(\cos (x))^{x^2}\right] & = \diff \left[e^{x^2\ln (\cos (x))}\right] \\[11pt]
    & = e^{x^2\ln (\cos (x))} \cdot \left(x^2 \cdot \dfrac{1}{\cos (x)} \cdot -(\sin (x)) + \ln (\cos (x)) \cdot 2x\right) \\[11pt]
    & = \boxed{(\cos (x))^{x^2}\left(2x\ln (\cos (x)) - x^2\tan (x)\right)}
\end{align*} 

\newpage

\textbf{\large{The Quotient Rule}}

\textbf{Ex 1.3.7: } $\diff \left[\dfrac{6x}{x^2 + 4}\right]$ \\[11pt]
\textbf{Sol 1.3.7: } \begin{align*}
    & U = 6x, \; \dfrac{dU}{dx} = 6 \\[11pt]
    & V = x^2 + 4, \; \dfrac{dV}{dx} = 2x \\[11pt]
    & \diff \left[\dfrac{6x}{x^2 + 4}\right] \begin{aligned}[t]
        & = \dfrac{\left(x^2 + 4\right) \cdot 6 - 6x \cdot 2x}{\left(x^2 + 4\right)^2} \\[11pt]
        & = \dfrac{6x^2 + 24 - 12x^2}{\left(x^2 + 4\right)} \\[11pt]
        & = \boxed{\dfrac{24 - 6x^2}{\left(x^2 + 4\right)}}
    \end{aligned}
\end{align*}

\textbf{Ex 1.3.8: } $\diff \left[\dfrac{x^2 + 2x - 3}{x - 4}\right]$ \\[11pt]
\textbf{Sol 1.3.8: } \begin{align*}
    & U = x^2 + 2x - 3, \; \dfrac{dU}{dx} = 2x + 2 \\[11pt]
    & V = x - 4, \; \dfrac{dV}{dx} = 1 \\[11pt]
    & \diff \left[\dfrac{x^2 + 2x - 3}{x - 4}\right] \begin{aligned}[t]
        & = \dfrac{(x - 4) \cdot (2x + 2) - \left(x^2 + 2x - 3\right) \cdot 1}{(x - 4)^2} \\[11pt]
        & = \dfrac{2x^2 - 6x - 8 - x^2 - 2x + 3}{(x - 4)^2} \\[11pt]
        & = \boxed{\dfrac{x^2 - 8x - 5}{(x - 4)^2}}
    \end{aligned}
\end{align*} 

\textbf{Ex 1.3.9: } $\diff \left[\dfrac{x^2 - 4x + 3}{2x^2 - 5x - 3}\right]$ \\[11pt]
\textbf{Sol 1.3.9: } Notice that this problem becomes much easier if we simplify before applying the Quotient Rule. \begin{align*}
    \diff \left[\dfrac{x^2 - 4x + 3}{2x^2 - 5x - 3}\right] & = \diff \left[\dfrac{(x - 1)(x - 3)}{(2x + 1)(x - 3)}\right] \\[11pt] 
    & = \diff \left[\dfrac{x - 1}{2x + 1}\right] \\[11pt]
    & U = x - 1, \; \dfrac{dU}{dx} = 1 \\[11pt]
    & V = 2x + 1, \; \dfrac{dV}{dx} = 2 \\[11pt]
    & \diff \left[\dfrac{x - 1}{2x + 1}\right] \begin{aligned}[t]
        & = \dfrac{(2x + 1) \cdot 1 - (x - 1) \cdot 2}{(2x + 1)^2} \\[11pt]
        & = \boxed{\dfrac{3}{(2x + 1)^2}}
    \end{aligned}
\end{align*}

\textbf{Ex 1.3.10: } $\diff \left[\dfrac{\cot (3x)}{x^2 + 1}\right]$ \\[11pt]
\textbf{Sol 1.3.10: } \begin{align*}
    & U = \cot (3x), \; \dfrac{dU}{dx} = -\csc^2 (3x) \cdot 3 = -3\csc^2 (3x)\\[11pt]
    & V = x^2 + 1, \; \dfrac{dV}{dx} = 2x \\[11pt]
    & \diff \left[\dfrac{\cot (3x)}{x^2 + 1}\right] \begin{aligned}[t]
        & = \dfrac{\left(x^2 + 1\right) \cdot \left(-3\csc^2 (3x)\right) - \cot (3x) \cdot 2x}{\left(x^2 + 1\right)^2} \\[11pt]
        & = \dfrac{-3x^2\csc^2 (3x) - 3\csc^2 (3x) - 2x\cot (3x)}{\left(x^2 + 1\right)^2} \\[11pt]
        & = \boxed{-\dfrac{\csc^{2}(3x)\left(3x^2 + 3\right) + 2x \cot(3x)}{\left(x^2 + 1\right)^2}}
    \end{aligned}
\end{align*}

As with the Product Rule, the difficulty with the Quotient Rule arises from the algebra needed to simplify our answer. \par

\textbf{Ex 1.3.11: } If $y = \dfrac{4x}{\sqrt{x^2 + 4}}$, find $\dfrac{dy}{dx}$ \\[11pt]
\textbf{Sol 1.3.11: } \begin{align*}
    & U = 4x, \; \dfrac{dU}{dx} = 4 \\[11pt]
    & V = \sqrt{x^2 + 4} = \left(x^2 + 4\right)^\frac{1}{2}, \; \dfrac{dV}{dx} = \dfrac{1}{2}\left(x^2 + 4\right)^{-\frac{1}{2}} \cdot 2x = \dfrac{2x}{2\sqrt{x^2 + 4}} \\[11pt]
    & \dfrac{dy}{dx} \begin{aligned}[t]
        & = \dfrac{\sqrt{x^2 + 4} \cdot 4 - 4x \cdot \dfrac{2x}{2\sqrt{x^2 + 4}}}{x^2 + 4} \\[11pt]
        & = \dfrac{\dfrac{4\left(x^2 + 4\right)}{\sqrt{x^2 + 4}} - \dfrac{4x^2}{\sqrt{x^2 + 4}}}{x^2 + 4} \\[11pt]
        & = \dfrac{4x^2 + 16 - 4x^2}{\left(x^2 + 4\right)^\frac{3}{2}} \\[11pt]
        & = \boxed{\dfrac{16}{\left(x^2 + 4\right)^\frac{3}{2}}}
    \end{aligned}
\end{align*}

\textbf{Ex 1.3.12: } Find the equation of the tangent line to $f(x) = \dfrac{x}{\sqrt{x^2 + 9}}$ at $x = -\sqrt{7}$. \\[11pt]
\textbf{Sol 1.3.12: } As we recall, for the equation of a line, we need a point and a slope. \begin{align*}
    & \text{The point: } f\left(-\sqrt{7}\right) \\[11pt]
    & \text{The slope is the derivative at the given x-value:} \\[11pt]
    & U = x, \; \dfrac{dU}{dx} = 1 \\[11pt]
    & V = 
\end{align*} 

