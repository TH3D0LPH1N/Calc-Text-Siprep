\textbf{\underline{\large{1.9: Logarithmic Differentiation}}} \par

With implicit differentiation and the Chain Rule, we learned some powerful tools for differentiating functions and relations. The Product and Quotient Rules also allowed us to take derivatives of certain functions that would otherwise be impossible to differentiate. Sometimes, however, with very complex functions, it becomes easier to manipulate an equation so that it is easier to take the derivative. This is where logarithmic differentiation comes in. \par

\begin{tcolorbox}[objective]
    \begin{center}
        OBJECTIVES \\[11pt]
    \end{center}
    Determine When It Is Appropriate to Use Logarithmic Differentiation. \\
    Use Logarithmic Differentiation to Take The Derivatives of Complicated Functions.
\end{tcolorbox}

Before we begin, it would be helpful to look at a few exponent and logarithm rules that we should recall from algebra and precalculus. 

\begin{center}
    \fbox{\fbox{\begin{minipage}{0.96\textwidth}
        \begin{align*}
            & a^xa^y = a^{x + y} && \log_a{x} + \log_a{y} = \log_a{xy} \\[11pt]
            & \dfrac{a^x}{a^y} = a^{x - y} && \log_a{x} - \log_a{y} = \log_a{\dfrac{x}{y}} \\[11pt]
            & \left(a^x\right)^y = a^{xy} && \log_a{x^n} = n\log_a{x} \\
        \end{align*}
    \end{minipage}}}
\end{center}

Since logarithms are exponents expressed in a different form, all of the above rules are derived from those of exponents.and you can see the corresponding exponential rule. Because of our algebraic rules, we can do whatever we want to both sides of an equation. In algebra, we usually used this to solve for a variable. In calculus, we can use this principle to make many derivative problems significantly easier. \par

\begin{tcolorbox}[example]
    \textbf{Ex 1.9.1: } Find the derivative of $y = \left(x^2 + 7x - 3\right)(\sin (x))$.
\end{tcolorbox}
\begin{tcolorbox}[solution]
    \textbf{Sol 1.9.1: } Traditionally, we would use the Product Rule to take the derivative of this function. \begin{align*}
        & \diff \left[y = \left(x^2 + 7x - 3\right)(\sin (x))\right] \\[11pt]
        & \deriv = \left(x^2 + 7x - 3\right)\cos (x) + (2x + 7)(\sin (x)) 
    \end{align*} 
    Obviously, this is a straightforward problem that can be easily done using the product rule. If, however, I took the natural log of both sides of the equation, I can achieve the same results, and never use the product rule. \begin{align*}
        & \ln y = \ln \left[\left(x^2 + 7x - 3\right)(\sin (x))\right] \\[11pt]
    \end{align*}
\end{tcolorbox}