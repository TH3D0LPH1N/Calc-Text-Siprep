\textbf{\underline{\large{1.4: Higher Order Derivatives}}} \par

What we've been calling the derivative is actually the first derivative. There can be successive uses of the derivative rules, and they have meanings other than the slope of the tangent line. In this section, we will explore the process of finding the higher-order derivatives. \par

\begin{tcolorbox}[definition]
    \begin{tabbing}
        \textit{Second Derivative} \= $\rightarrow$ \= Definition: The derivative of the derivative.
    \end{tabbing}
\end{tcolorbox}

Just as with the first derivative, there are several symbols for the second derivative. \par

\begin{center}  
    \fbox{\fbox{\begin{minipage}{0.96 \textwidth}
        \vspace{11pt}
        \begin{center}
            Higher Order Derivative Symbols
            \\[11pt]
            $\deriv[2] = \text{``d squared y, d - x squared''} \rightarrow \dfrac{d^3y}{dx^3}, \; \dfrac{d^4y}{dx^4} \dots \dfrac{d^ny}{dx^n}$ \\[11pt]
            $\dfrac{d^2}{dx^2} = \text{``d squared, d - x squared''} \rightarrow \dfrac{d^3}{dx^3}, \; \dfrac{d^4}{dx^4} \dots \dfrac{d^n}{dx^n}$ \\[11pt]
            $f''(x) = \text{``f double prime of x''} \rightarrow f'''(x), \; f^{IV}(x) \dots f^{n}(x)$ \\[11pt]
            $y'' = \text{``y double prime''}$ 
        \end{center} 
        \vspace{11pt}
    \end{minipage}}}
\end{center} 

\begin{tcolorbox}[objective]
    \begin{center}
        OBJECTIVES \\[11pt]
    \end{center}
    Find Higher Order Derivatives.
\end{tcolorbox} \vspace{11pt}

\begin{tcolorbox}[example]
    \textbf{Ex 1.4.1: } $\dfrac{d^2}{dx^2} \left[x^4 - 7x^3 - 3x^2 + 2x - 5\right]$
\end{tcolorbox}
\begin{tcolorbox}[solution]
    \textbf{Sol 1.4.1: } \begin{align*}
        \dfrac{d^2}{dx^2} \left[x^4 - 7x^3 - 3x^2 + 2x - 5\right] &= \diff \left[\diff \left[x^4 - 7x^3 - 3x^2 + 2x - 5\right]\right] \\[11pt]
        & = \diff \left[4x^3 - 21x^2 - 6x + 2\right] \\[11pt]
        & = \boxed{12x^2 - 42x - 6}
    \end{align*}
\end{tcolorbox} \vspace{11pt}
 
\begin{tcolorbox}[example]
    \textbf{Ex 1.4.2: } Find $\dfrac{d^3y}{dx^3}$ if $y = \sin (3x)$.
\end{tcolorbox} 
\begin{tcolorbox}[solution]
    \textbf{Sol 1.4.2: } \begin{align*}
        & y = \sin (3x) \\[11pt]
        & \deriv = \cos (3x) \cdot 3 = 3\cos (3x) \\[11pt]
        & \deriv[2] = 3(-\sin (3x)) \cdot 3 = -9\sin (3x) \\[11pt]
        & \dfrac{d^3y}{dx^3} = -9\cos (3x) \cdot 3 = \boxed{-27\cos (3x)}
    \end{align*}
\end{tcolorbox}

More complicated functions, in particular \hyperlink{Composite Function}{composite functions}, have a more complicated process. When the Chain Rule is applied, the result often includes a product or a quotient. Therefore, the second derivative will require the Product or Quotient Rules, as well as possibly the Chain Rule. \par

\begin{tcolorbox}[example]
    \textbf{Ex 1.4.3: } $y = e^{3x^2}$, find $\dfrac{d^y}{dx^2}$. 
\end{tcolorbox}
\begin{tcolorbox}[solution]
    \textbf{Sol 1.4.3: }\begin{align*}
        & \deriv = e^{3x^2} \cdot 6x = 6xe^{3x^2} \\[11pt]
        & \deriv[2] \begin{aligned}[t]
            & = 6x\left(e^{3x^2} \cdot 6x\right) + e^{3x^2} \cdot 6 \\[11pt]
            & = 36x^2e^{3x^2} + 6e^{3x^2} \\[11pt]
            & = \boxed{6e^{3x^2}\left(6x^2 + 1\right)}
        \end{aligned}
    \end{align*}
\end{tcolorbox} \vspace{11pt}

\begin{tcolorbox}[example]
    \textbf{Ex 1.4.4: } $y = \sin^3 (x)$, find $y''$.
\end{tcolorbox}
\begin{tcolorbox}[solution] 
    \textbf{Sol 1.4.4: } \begin{align*}
        & y' = 3\sin^2 (x) \cdot \cos (x) \\[11pt]
        & y'' \begin{aligned}[t]
            & = 3\sin^2 (x)(-\sin (x)) + \cos (x)(6\sin (x) \cdot \cos (x)) \\[11pt]
            & = \boxed{3 \sin (x)\left(2\cos^2(x) - \sin^2(x)\right)}
        \end{aligned}
    \end{align*}
\end{tcolorbox} \vspace{11pt}

\begin{tcolorbox}[example]
    \textbf{Ex 1.4.5: } $f(x) = \ln \left(x^2 + 3x - 1\right)$, find $f''(x)$.
\end{tcolorbox}
\begin{tcolorbox}[solution]
    \textbf{Sol 1.4.5: } \begin{align*}
        & f'(x) = \dfrac{1}{x^2 + 3x - 1}\left(2x + 3\right) = \dfrac{2x + 3}{x^2 + 3x - 1} \\[11pt]
        & f''(x) \begin{aligned}[t]
            & = \dfrac{\left(x^2 + 3x - 1\right)(2) - (2x + 3)(2x + 3)}{\left(x^2 + 3x - 1\right)^2} \\[11pt]
            & = \dfrac{\left(2x^2 + 6x - 2\right) - \left(4x^2 + 12x + 9\right)}{\left(x^2 + 3x - 1\right)^2} \\[11pt]
            & = \boxed{-\dfrac{2x^2 - 6x - 11}{\left(x^2 + 3x - 1\right)^2}}
        \end{aligned}
    \end{align*}
\end{tcolorbox} \vspace{11pt}

\begin{tcolorbox}[example]
    \textbf{Ex 1.4.6: } $g(x) = \sqrt{4x^2 + 1}$, find $g''(x)$
\end{tcolorbox}
\begin{tcolorbox}[solution]
    \textbf{Sol 1.4.6: } \begin{align*}
        & g'(x) = \dfrac{1}{2}\left(4x^2 + 1\right)^{-\frac{1}{2}}(8x) = \dfrac{4x}{\left(4x^2 + 1\right)^{\frac{1}{2}}} \\[11pt]
        & g''(x) \begin{aligned}[t]
            & = \dfrac{\left(4x^2 + 1\right)^{\frac{1}{2}}(4) - (4x)\left[\dfrac{1}{2}\left(4x^2 + 1\right)^{-\frac{1}{2}}(8x)\right]}{\left[\left(4x^2 + 1\right)^{\frac{1}{2}}\right]^2} \\[11pt]
            & = \dfrac{{\left(4x^2 + 1\right)^{\frac{1}{2}}}(4) - \dfrac{16x^2}{\left(4x^2 + 1\right)^{\frac{1}{2}}}}{\left(4x^2 + 1\right)} \\[11pt]
            & = \dfrac{\left(4x^2 + 1\right)(4) - 16x^2}{\left(4x^2 + 1\right)^{\frac{3}{2}}} \\[11pt]
            & = \boxed{\dfrac{4}{\left(4x^2 + 1\right)^{\frac{3}{2}}}}
        \end{aligned}
    \end{align*}
\end{tcolorbox}

\newpage

\textbf{\large{1.4 Free Response Homework}} \par

Find the second derivatives of the given functions. Simplify where possible. \par

\twoquestion{1. $f(x) = x^5 + 6x^2 - 7x$}{2. $h(x) = 5x^4 + 9x^3 - 4x^2 + x - 8$} \\[11pt]
\twoquestion{3. $y = \left(x^3 + 1\right)^{\frac{2}{3}}$}{4. $H(t) = \tan (3t)$} \\[11pt]
\twoquestion{5. $g(t) = t^3e^{5t}$}{6. $y = e^{3x^2}$} \\[11pt] 
\twoquestion{7. $y = \sin^4(x)$}{8. $f(t) = t\cos (t)$} \\[11pt]
\twoquestion{9. $y = -\dfrac{4x}{x^2 + 4}$}{10. $y = \dfrac{x^2 - 1}{x^2 - 4}$} \\[11pt]
\twoquestion{11. $f(x) = x\sqrt{8 - x^2}$}{12. $y = \dfrac{1}{2}x + \sin (x)$} \\[11pt]
\twoquestion{13. $g(t) = te^{-t}$}{14. $y = e^{-x^2}$} \\[11pt]
\twoquestion{15. $y = \dfrac{x}{x^2 - 9}$}{16. $B(x) = 2x - x^{\frac{2}{3}}$} \\[11pt]
\twoquestion{17. $y = x^3 + x^2 - 7x + 15$}{19. $y = 3x^4 - 20x^3 + 42x^2 - 36x + 16$} \\[11pt]

Complete the following: \par

\twoquestion{21. $y = \cos \left(x^2\right)$, find $y''$}{22. $y = \tan^2 (x)$, find $y''$} \\[11pt]
\twoquestion{23. $y = \sec (3x)$, find $\deriv[2]$}{24. $y = xe^{2x}$, find $\deriv[2]$} \\[11pt]
\twoquestion{25. $f(x) = \ln \left(x^2 + 3\right)$, find $f''(x)$}{25. $g(x) = \ln \left(x^2 - 4x + 4\right)$, find $g''(x)$} \\[11pt]
\twoquestion{27. $h(x) = \sqrt{x^2 + 5}$, find $h''(x)$}{28. $F(x) = \sqrt{3x^2 - 2x + 1}$, find $F''(x)$} \\[11pt]
\twoquestion{29. $y = \dfrac{x^2 - 3}{x^2 - 10}$, find $\deriv[2]$}{30. $y = \dfrac{3x + 3}{x^3 + 1}$, find $\deriv[2]$} \\[11pt]

\textbf{\large{1.4 Multiple Choice Homework}} \par

\begin{questions}
    \question If $f$ and $g$ are twice differentiable and if $h(x) = g\left(f(x)\right)$, then $h''(x) = $ \\

    \begin{oneparchoices}
        \choice $g''\left(f(x)\right)$ 
        \choice $g''\left(f(x)\right)f''(x)$ 
        \choice $g''\left(f(x)\right)\left[f'(x)\right]^2$ \\[11pt]
        \makebox[0.07\textwidth] \choice $g'\left(f(x)\right)\left[f'(x)\right]^2 + f'(x)\left(f''(x)\right)$ 
        \makebox[0.13\textwidth] \choice $g'\left(f(x)\right)f''(x) + \left[f'(x)\right]^2g''\left(f(x)\right)$ 
    \end{oneparchoices} \par \horizontalline

    \question Find $\deriv[2]$ if $y = \dfrac{x + 2}{x - 3}$ \\

    \begin{oneparchoices}
        \choice $-\dfrac{2}{(x - 3)^2}$
        \choice $0$
        \choice $\dfrac{10}{(x - 3)^3}$
        \choice $\dfrac{2}{(x - 3)^2}$
        \choice None of these
    \end{oneparchoices} \par \horizontalline

    \question If $y = \ln (\cos (x))$ and $0 \leq x \leq \pi$, then $\deriv[2]$ is \\

    \begin{oneparchoices}
        \choice $-\tan (x)$
        \choice $-\sec^2 (x)$
        \choice $\tan (x)$
        \choice $\sec^2 (x)$
        \choice $\sec (x)\tan (x)$
    \end{oneparchoices} \par \horizontalline

    \question If $y = \ln \left(x^2 + 4\right)$, then $\deriv[2]$ is \\

    \begin{oneparchoices}
        \choice $\dfrac{1}{x^2 + 4}$ 
        \choice $\dfrac{2x}{x^2 + 4}$
        \choice $\dfrac{-2x^2 + 8}{x^2 + 4}$
        \choice $\dfrac{2x}{\left(x^2 + 4\right)^2}$
        \choice $\dfrac{-2x^2 + 8}{\left(x^2 + 4\right)^2}$
    \end{oneparchoices} \par \horizontalline

    \question If $y = e^{x^2}$, then $\deriv[2] = $ \\

    \begin{oneparchoices}
        \choice $e^{x^2}$
        \choice $2e^{x^2}\left(2x^2 + 1\right)$
        \choice $2xe^{x^2}$
        \choice $4x^2e^{x^2}$
        \choice $2e^{x^2}\left(2x^2 - 1\right)$
    \end{oneparchoices} \par \horizontalline

    \question If $h(t) = \ln \left(t^2 + 1\right)$, then $h''(-1) = $ \\

    \begin{oneparchoices}
        \choice $\ln 2$
        \choice $0$
        \choice $-1$
        \choice $-2$
        \choice $DNE$
    \end{oneparchoices} \par \horizontalline

    \question If $y = \sin \left(e^x\right)$, then $\deriv[2] = $ \\

    \begin{oneparchoices}
        \choice $\cos \left(e^x\right)$
        \choice $e^x \cos \left(e^x\right)$
        \choice $e^x\sin \left(e^x\right) + e^x\cos \left(e^x\right)$ \\[11pt]
        \makebox[0.12\textwidth] \choice $-e^x\sin \left(e^x\right) + e^x\cos \left(e^x\right)$
        \makebox[0.20\textwidth]\choice $-e^x\left(e^x\sin \left(e^x\right) - \cos \left(e^x\right)\right)$
    \end{oneparchoices} \par \horizontalline
\end{questions}