\textbf{\underline{\large{Chapter 2 Overview: Anti-Derivatives}}} \par

As noted in the introduction, Calculus is essentially comprised of four operations: \begin{itemize}
    \item Limits
    \item Derivatives
    \item Indefinite Integrals (Or Anti-Derivatives)
    \item Definite Integrals
\end{itemize}

As mentioned above, there are two types of integrals --- the definite integral and the indefinite integral. The definite integral was explored first as a way to determine the area bounded by a curve, rather than bounded by a polygon. The summation of infinite rectangles is \begin{align*}
    A = \sum_{i = 1}^n f\left(x_i\right) \cdot \Delta x,
\end{align*}
and the symbol \begin{align*}
    \int_a^b f(x) \, dx
\end{align*}
is the exact amount, with $\int$ being an elongated and stylized $s$ for ``sum''. \par

Newton and Leibnitz made the connection between the definite integral and the antiderivative, showing that the process of reversing the derivative results in an infinite summation. The antiderivative and indefinite integral are inverses of each other, just as squares and square roots or exponential and log functions. In this chapter, we will consider how to reverse the differentiation process. In a later chapter, we will dive deeper into the definite integral. Let's start by reviewing our derivative rules, as they will be necessary for us to take the antiderivative. \par

\begin{center}
    \fbox{\fbox{\begin{minipage}{0.96\textwidth}
        \vspace{11pt}
        \textbf{You must know the derivative rules in order to know the antiderivative rules!}
        \vspace{11pt}
        \begin{align*}
            & \text{The Power Rule: } \diff \left[u^n\right] = nu^{n - 1} \dfrac{du}{dx} \\[11pt]
            & \text{The Product Rule: } \diff \left[u \cdot v\right] = u \cdot \dfrac{dv}{dx} + v \cdot \dfrac{du}{dx} \\[11pt]
            & \text{The Quotient Rule: } \diff \left[\dfrac{u(x)}{v(x)}\right] = \dfrac{v \cdot \dfrac{du}{dx} - u \cdot \dfrac{dv}{dx}}{v^2} \\[11pt]
            & \text{The Chain Rule: } \diff \left[f(g(x))\right] = f'(g(x)) \cdot g'(x)
        \end{align*}
        \begin{align*}
            & \diff[\sin u] = (\cos u) \dfrac{du}{dx} && \diff[\csc u] = (-\csc u \cot u) \dfrac{du}{dx} \\[11pt] % Line 1
            & \diff[\cos u] = (-\sin u) \dfrac{du}{dx} && \diff[\sec u] = (\sec u \tan u) \dfrac{du}{dx} \\[11pt] % Line 2
            & \diff[\tan u] = \left(\sec^2 u\right) \dfrac{du}{dx} && \diff[\cot u] = \left(-\csc^2 u\right) \dfrac{du}{dx} \\[11pt] % Line 3
            & \diff\left[e^u\right] = \left(e^u\right) \dfrac{du}{dx} && \diff[\ln{u}] = \left(\dfrac{1}{u}\right) \dfrac{du}{dx}\\[11pt] % Line 4
            & \diff\left[a^u\right] = \left(a^u \cdot \ln{u}\right) \dfrac{du}{dx} && \diff\left[\log_a{u}\right] = \left(\dfrac{1}{u \cdot \ln{a}}\right) \dfrac{du}{dx} \\[11pt] % Line 5
            & \diff\left[\sin^{-1} u\right] = \left(\dfrac{1}{\sqrt{1 - u^2}} \right) \dfrac{du}{dx} && \diff\left[\csc^{-1} u\right] = \left(\dfrac{-1}{|u|\sqrt{u^2 - 1}}\right) \dfrac{du}{dx} \\[11pt] % Line 6
            & \diff\left[\cos^{-1} u\right] = \left(\dfrac{-1}{\sqrt{1 - u^2}} \right) \dfrac{du}{dx} && \diff\left[\sec^{-1} u\right] = \left(\dfrac{1}{|u|\sqrt{u^2 - 1}}\right) \dfrac{du}{dx} \\[11pt] % Line 7 
            & \diff\left[\tan^{-1} u\right] = \left(\dfrac{1}{u^2+ 1} \right) \dfrac{du}{dx} && \diff\left[\cot^{-1} u\right] = \left(\dfrac{-1}{u^2 + 1}\right) \dfrac{du}{dx} \\[11pt] % Line 8
        \end{align*}
    \end{minipage}}}
\end{center}