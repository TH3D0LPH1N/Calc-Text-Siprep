\pagecolor{gray!10}%
Full Name: \hfill AP Calculus BC \\[22pt]
Date: \hfill Chapter 2 Practice Test 1 \\[22pt]
\textbf{Multiple Choice Section} \hfill \textbf{20 Minutes; No Calculator} \\[11pt]

\begin{center}
    \textbf{Show All Work}
\end{center} \vspace{11pt}

\begin{questions}
    %Question 1
    \question Which of the following is \textbf{true}? \\

    \begin{oneparchoices}
        \choice $\int \left(5^x + 2e^{-x}\right) \, dx = \dfrac{5^x}{\ln 5} + 2e^{-x} + C$
        \choice $\int \cot^3 (x)\csc^2 (x) \, dx = \dfrac{1}{4}\cot^4 (x) + C$ \\[11pt]
        \makebox[0.035\textwidth] \choice $\int (x - 1)\sqrt{x} \, dx = \dfrac{2}{3}x^{\frac{3}{2}} - \dfrac{1}{2}x^{\frac{1}{2}} + C$
        \makebox[0.15\textwidth] \choice $\int \dfrac{\sin \sqrt{x}}{\sqrt{x}} \, dx = -2\cos \left(\sqrt{x}\right) + C$
    \end{oneparchoices} \par \horizontalline

    %Question 2
    \question $\int \dfrac{x}{x^2 - 4} \, dx = $ \\
    
    \begin{oneparchoices}
        \choice $-\dfrac{1}{4\left(x^2 - 4\right)^2} + C$
        \choice $-\dfrac{1}{2\left(x^2 - 4\right)} + C$
        \choice $\dfrac{1}{2}\ln |x^2 - 4| + C$ \\[11pt]
        \makebox[0.21\textwidth] \choice $2\ln |x^2 - 4| + C$ 
        \makebox[0.23\textwidth] \choice $\dfrac{1}{2}\arctan \left(\dfrac{x}{2}\right) + C$
    \end{oneparchoices} \par \horizontalline

    \newpage

    %Question 3
    \question Which of the following slope fields presents $\deriv = y^2 + 2x^2$? \\

    \begin{oneparchoices}
        \choice \begin{tikzpicture}[declare function={f(\x,\y)= (\y)^2 + 2 * (\x)^2;},scale=0.6] 
            %                       Function goes here ^^^^^ Use \x and \y.
            %                       Change scale to make bigger or smaller.
            \def\xmin{-4.001}       \def\xmax{4.001}    % Set domain and range for
            \def\ymin{-4.001}    \def\ymax{4.001} % the slopes.  
            % ymin and ymax being non-integer can help with division by zero errors.
            \def\res{2} % resolution of the slope field
            \def\size{2mm} % size of each slope in mm
            %%%%%%%%%% do not change anything below this %%%%%%%%%%
            \pgfmathsetmacro{\nx}{(\xmax-\xmin) * \res} 
            \pgfmathsetmacro{\ny}{(\ymax-\ymin) * \res} 
            \draw[help lines, color=gray!50] (\xmin -.5,\ymin -.5) grid (\xmax +.5,\ymax +.5);
            \pgfmathsetmacro{\hx}{(\xmax-\xmin)/\nx}
            \pgfmathsetmacro{\hy}{(\ymax-\ymin)/\ny}
            \foreach \i in {0,...,\nx}
            \foreach \j in {0,...,\ny}{
                    \pgfmathsetmacro{\yprime}{f({\xmin+\i*\hx},{\ymin+\j*\hy})}
                    \draw[thick, shift={({\xmin+\i*(\xmax-\xmin)/\nx},{\ymin+\j*(\ymax-\ymin)/\ny})}]
                    ($(0,0)!\size!(-.1,-.1*\yprime)$)--($(0,0)!\size!(.1,.1*\yprime)$);
                }
            \draw[->] (\xmin-.5,0)--(\xmax+.5,0) node[below right] {\(x\)};
            \draw[->] (0,\ymin-.5)--(0,\ymax+.5) node[above left] {\(y\)};
            %%%%%%%%%%%%% and above this %%%%%%%%%%%%%%%%
            %
            % Uncomment below two lines to include a solution.
            % The function is where FUNCTION goes and is in terms of \x.
            % e.g. "(\x)^2/(\x+1)" 
            % if you want to use trig functions, wrap your argument in "deg".
            % e.g. "sin(deg(\x))%
            %
            % \clip (\xmin -.5,\ymin -.5) rectangle (\xmax +.5,\ymax +.5);
            % \draw[domain=\xmin:\xmax, smooth, variable=\x, red, ultra thick] plot ({\x}, {-cos(deg(\x)) });
            % 
            % Uncomment below line to draw a point at (POINT) e,g, (2,1)
            %
            %\filldraw (POINT) circle (0.1);
        \end{tikzpicture}
        \choice \begin{tikzpicture}[declare function={f(\x,\y)= sin(deg(\x));},scale=0.6] 
            %                       Function goes here ^^^^^ Use \x and \y.
            %                       Change scale to make bigger or smaller.
            \def\xmin{-4.001}       \def\xmax{4.001}    % Set domain and range for
            \def\ymin{-4.001}    \def\ymax{4.001} % the slopes.  
            % ymin and ymax being non-integer can help with division by zero errors.
            \def\res{2} % resolution of the slope field
            \def\size{2mm} % size of each slope in mm
            %%%%%%%%%% do not change anything below this %%%%%%%%%%
            \pgfmathsetmacro{\nx}{(\xmax-\xmin) * \res} 
            \pgfmathsetmacro{\ny}{(\ymax-\ymin) * \res} 
            \draw[help lines, color=gray!50] (\xmin -.5,\ymin -.5) grid (\xmax +.5,\ymax +.5);
            \pgfmathsetmacro{\hx}{(\xmax-\xmin)/\nx}
            \pgfmathsetmacro{\hy}{(\ymax-\ymin)/\ny}
            \foreach \i in {0,...,\nx}
            \foreach \j in {0,...,\ny}{
                    \pgfmathsetmacro{\yprime}{f({\xmin+\i*\hx},{\ymin+\j*\hy})}
                    \draw[thick, shift={({\xmin+\i*(\xmax-\xmin)/\nx},{\ymin+\j*(\ymax-\ymin)/\ny})}]
                    ($(0,0)!\size!(-.1,-.1*\yprime)$)--($(0,0)!\size!(.1,.1*\yprime)$);
                }
            \draw[->] (\xmin-.5,0)--(\xmax+.5,0) node[below right] {\(x\)};
            \draw[->] (0,\ymin-.5)--(0,\ymax+.5) node[above left] {\(y\)};
            %%%%%%%%%%%%% and above this %%%%%%%%%%%%%%%%
            %
            % Uncomment below two lines to include a solution.
            % The function is where FUNCTION goes and is in terms of \x.
            % e.g. "(\x)^2/(\x+1)" 
            % if you want to use trig functions, wrap your argument in "deg".
            % e.g. "sin(deg(\x))%
            %
            % \clip (\xmin -.5,\ymin -.5) rectangle (\xmax +.5,\ymax +.5);
            % \draw[domain=\xmin:\xmax, smooth, variable=\x, red, ultra thick] plot ({\x}, {-cos(deg(\x)) });
            % 
            % Uncomment below line to draw a point at (POINT) e,g, (2,1)
            %
            %\filldraw (POINT) circle (0.1);
        \end{tikzpicture} \\[11pt]
        \makebox[0.035\textwidth] \choice \begin{tikzpicture}[declare function={f(\x,\y)=0.25 * (\x)^2 - 0.25 * (\y)^2;},scale=0.6] 
            %                       Function goes here ^^^^^ Use \x and \y.
            %                       Change scale to make bigger or smaller.
            \def\xmin{-4.001}       \def\xmax{4.001}    % Set domain and range for
            \def\ymin{-4.001}    \def\ymax{4.001} % the slopes.  
            % ymin and ymax being non-integer can help with division by zero errors.
            \def\res{2} % resolution of the slope field
            \def\size{2mm} % size of each slope in mm
            %%%%%%%%%% do not change anything below this %%%%%%%%%%
            \pgfmathsetmacro{\nx}{(\xmax-\xmin) * \res} 
            \pgfmathsetmacro{\ny}{(\ymax-\ymin) * \res} 
            \draw[help lines, color=gray!50] (\xmin -.5,\ymin -.5) grid (\xmax +.5,\ymax +.5);
            \pgfmathsetmacro{\hx}{(\xmax-\xmin)/\nx}
            \pgfmathsetmacro{\hy}{(\ymax-\ymin)/\ny}
            \foreach \i in {0,...,\nx}
            \foreach \j in {0,...,\ny}{
                    \pgfmathsetmacro{\yprime}{f({\xmin+\i*\hx},{\ymin+\j*\hy})}
                    \draw[thick, shift={({\xmin+\i*(\xmax-\xmin)/\nx},{\ymin+\j*(\ymax-\ymin)/\ny})}]
                    ($(0,0)!\size!(-.1,-.1*\yprime)$)--($(0,0)!\size!(.1,.1*\yprime)$);
                }
            \draw[->] (\xmin-.5,0)--(\xmax+.5,0) node[below right] {\(x\)};
            \draw[->] (0,\ymin-.5)--(0,\ymax+.5) node[above left] {\(y\)};
            %%%%%%%%%%%%% and above this %%%%%%%%%%%%%%%%
            %
            % Uncomment below two lines to include a solution.
            % The function is where FUNCTION goes and is in terms of \x.
            % e.g. "(\x)^2/(\x+1)" 
            % if you want to use trig functions, wrap your argument in "deg".
            % e.g. "sin(deg(\x))%
            %
            % \clip (\xmin -.5,\ymin -.5) rectangle (\xmax +.5,\ymax +.5);
            % \draw[domain=\xmin:\xmax, smooth, variable=\x, red, ultra thick] plot ({\x}, {-cos(deg(\x)) });
            % 
            % Uncomment below line to draw a point at (POINT) e,g, (2,1)
            %
            %\filldraw (POINT) circle (0.1);
        \end{tikzpicture} 
        \choice \begin{tikzpicture}[declare function={f(\x,\y)=0.25 * (\y)^2 - 0.25 * (\x)^2;},scale=0.6] 
            %                       Function goes here ^^^^^ Use \x and \y.
            %                       Change scale to make bigger or smaller.
            \def\xmin{-4}       \def\xmax{4}    % Set domain and range for
            \def\ymin{-4}    \def\ymax{4} % the slopes.  
            % ymin and ymax being non-integer can help with division by zero errors.
            \def\res{2} % resolution of the slope field
            \def\size{2mm} % size of each slope in mm
            %%%%%%%%%% do not change anything below this %%%%%%%%%%
            \pgfmathsetmacro{\nx}{(\xmax-\xmin) * \res} 
            \pgfmathsetmacro{\ny}{(\ymax-\ymin) * \res} 
            \draw[help lines, color=gray!50] (\xmin -.5,\ymin -.5) grid (\xmax +.5,\ymax +.5);
            \pgfmathsetmacro{\hx}{(\xmax-\xmin)/\nx}
            \pgfmathsetmacro{\hy}{(\ymax-\ymin)/\ny}
            \foreach \i in {0,...,\nx}
            \foreach \j in {0,...,\ny}{
                    \pgfmathsetmacro{\yprime}{f({\xmin+\i*\hx},{\ymin+\j*\hy})}
                    \draw[thick, shift={({\xmin+\i*(\xmax-\xmin)/\nx},{\ymin+\j*(\ymax-\ymin)/\ny})}]
                    ($(0,0)!\size!(-.1,-.1*\yprime)$)--($(0,0)!\size!(.1,.1*\yprime)$);
                }
            \draw[->] (\xmin-.5,0)--(\xmax+.5,0) node[below right] {\(x\)};
            \draw[->] (0,\ymin-.5)--(0,\ymax+.5) node[above left] {\(y\)};
            %%%%%%%%%%%%% and above this %%%%%%%%%%%%%%%%
            %
            % Uncomment below two lines to include a solution.
            % The function is where FUNCTION goes and is in terms of \x.
            % e.g. "(\x)^2/(\x+1)" 
            % if you want to use trig functions, wrap your argument in "deg".
            % e.g. "sin(deg(\x))%
            %
            % \clip (\xmin -.5,\ymin -.5) rectangle (\xmax +.5,\ymax +.5);
            % \draw[domain=\xmin:\xmax, smooth, variable=\x, red, ultra thick] plot ({\x}, {-cos(deg(\x)) });
            %
            % Uncomment below line to draw a point at (POINT) e,g, (2,1)
            %
            %\filldraw (POINT) circle (0.1);
        \end{tikzpicture}
    \end{oneparchoices} \par \horizontalline

    \newpage

    %Question 4
    \question Which of the following equations corresponds to the solution to the slope field shown in the figure below? \begin{center}
        \begin{tikzpicture}[declare function={f(\x,\y)= 1/(-sin(deg(\x))) * 1/(tan(deg(\x)));},scale=0.8] 
            %                       Function goes here ^^^^^ Use \x and \y.
            %                       Change scale to make bigger or smaller.
            \def\xmin{-4.001}       \def\xmax{4.001}    % Set domain and range for
            \def\ymin{-4.001}    \def\ymax{4.001} % the slopes.  
            % ymin and ymax being non-integer can help with division by zero errors.
            \def\res{2} % resolution of the slope field
            \def\size{2mm} % size of each slope in mm
            %%%%%%%%%% do not change anything below this %%%%%%%%%%
            \pgfmathsetmacro{\nx}{(\xmax-\xmin) * \res} 
            \pgfmathsetmacro{\ny}{(\ymax-\ymin) * \res} 
            \draw[help lines, color=gray!50] (\xmin -.5,\ymin -.5) grid (\xmax +.5,\ymax +.5);
            \pgfmathsetmacro{\hx}{(\xmax-\xmin)/\nx}
            \pgfmathsetmacro{\hy}{(\ymax-\ymin)/\ny}
            \foreach \i in {0,1,2,3,4,5,6,7,9,10,11,12,13,14,15,16}
            \foreach \j in {0,...,\ny}{
                    \pgfmathsetmacro{\yprime}{f({\xmin+\i*\hx},{\ymin+\j*\hy})}
                    \draw[thick, shift={({\xmin+\i*(\xmax-\xmin)/\nx},{\ymin+\j*(\ymax-\ymin)/\ny})}]
                    ($(0,0)!\size!(-.1,-.1*\yprime)$)--($(0,0)!\size!(.1,.1*\yprime)$);
                }
            \draw[->] (\xmin-.5,0)--(\xmax+.5,0) node[below right] {\(x\)};
            \draw[->] (0,\ymin-.5)--(0,\ymax+.5) node[above left] {\(y\)};
            %%%%%%%%%%%%% and above this %%%%%%%%%%%%%%%%
            %
            % Uncomment below two lines to include a solution.
            % The function is where FUNCTION goes and is in terms of \x.
            % e.g. "(\x)^2/(\x+1)" 
            % if you want to use trig functions, wrap your argument in "deg".
            % e.g. "sin(deg(\x))%
            %
            % \clip (\xmin -.5,\ymin -.5) rectangle (\xmax +.5,\ymax +.5);
            % \draw[domain=\xmin:\xmax, smooth, variable=\x, red, ultra thick] plot ({\x}, {-cos(deg(\x)) });
            % 
            % Uncomment below line to draw a point at (POINT) e,g, (2,1)
            %
            %\filldraw (POINT) circle (0.1);
        \end{tikzpicture} 
    \end{center} \vspace{11pt}

    \begin{oneparchoices}
        \choice $y = x^4 - 8x^2$
        \choice $y = 8x^2 - 4x^4$
        \choice $y = \csc (x)$
        \choice $y = -\sec (x)$
    \end{oneparchoices} \par \horizontalline

    %Question 5
    \question Which of the following is a solution to the differential equation $\deriv = \dfrac{x - 1}{y}$ with the initial condition $y(0) = -2$? \\
    
    \begin{oneparchoices}
        \choice $y  = -2e^{x^2 - 2x}$
        \choice $y = -2 + e^{x^2 - 2}$
        \choice $y = \sqrt{x^2 - 2x - 4}$ \\[11pt]
        \makebox[0.17\textwidth] \choice $y = -\sqrt{x^2 - 2x + 4}$
        \makebox[0.2\textwidth] \choice $y = -\sqrt{x^2 - 2x - 4}$
    \end{oneparchoices} \par \horizontalline

    %Question 6
    \question $\int \dfrac{4y^3 - 2y^2 - 5y}{\sqrt{y}} \, dy = $ \\

    \begin{oneparchoices}
        \choice $\left(y^4 - \dfrac{2}{3}y^3 - \dfrac{5}{2}y^2\right)2y^{\frac{1}{2}}$ 
        \makebox[0.3\textwidth] \choice $4y^{\frac{5}{2}} - 2y^{\frac{3}{2}} - 5y^{\frac{1}{2}} + C$ \\[11pt]
        \makebox[0.035\textwidth] \choice $\dfrac{8}{7}y^{\frac{7}{2}} - \dfrac{4}{5}y^{\frac{5}{2}} - \dfrac{10}{3}y^{\frac{3}{2}} + C$
        \makebox[0.285\textwidth] \choice $10y^{\frac{3}{2}} - 3y^{\frac{1}{2}} - \dfrac{5}{2}y^{-\frac{1}{2}} + C$
    \end{oneparchoices} \par \horizontalline

    %Question 7
    \question Identify the first mistake (if any) in this process: \begin{align*}
        & \textbf{Problem:} && \deriv = x - xy\\[11pt]
        & \text{Step 1:} && \dfrac{1}{1 - y} \, dy = x \, dx \\[5.5pt]
        & \text{Step 2:} && -\ln |1 - y| = \dfrac{1}{2}x^2 + C \\[5.5pt]
        & \text{Step 3:} && |1 - y| = e^{-\frac{1}{2}x^2 + C} \\[5.5pt]
        & \text{Step 4:} && y = 1 + Ke^{\frac{1}{2}x^2}
    \end{align*}

    \begin{oneparchoices}
        \choice Step 1
        \choice Step 2
        \choice Step 3
        \choice Step 4
        \choice No mistake
    \end{oneparchoices} \par \horizontalline

    %Question 8
    \question An object moves with velocity $v(t) = \sec^2 (2t)$. It is known that the particle's position at time $0$ is $2$. What is the particle's position function? \\

    \begin{oneparchoices}
        \choice $s(t) = \tan (2t) + 2$
        \choice $s(t) = \dfrac{1}{2}\tan (2t) + 2$
        \choice $s(t) = \sec^2 (2t)\tan^2 (2t) + 2$ \\[11pt]
        \makebox[0.18\textwidth] \choice $s(t) = \ln |\sec (2t)| + 2$
        \makebox[0.15\textwidth] \choice $s(t) = \dfrac{1}{2} \ln |\sec (2t)| + 2$
    \end{oneparchoices} \par \horizontalline
\end{questions}

\newpage
\stepcounter{sectioncount}

\textbf{Free Response Section} \hfill \textbf{35 Minutes; No Calculator} \\[11pt]

\begin{center}
    \textbf{Show All Work}
\end{center}
\vspace{11pt}

\begin{questions}
    \question Compute the following antiderivatives. \\

    \begin{parts}
        \part $\int \left(2x^5 + 2^x - \dfrac{17}{\sqrt[5]{x^2}} - \dfrac{1}{5x^2}\right) \, dx$

        \vfill

        \parthline

        \part $\int e^{5x}\csc \left(e^{5x}\right) \, dx$

        \vfill

        \parthline

        \newpage

        \part $\int \left(\cos (3x) + \cos^2 (5x) + \sin (7x)\sqrt{\cos (7x)}\right) \, dx$

        \vfill
    \end{parts} 

    \horizontalline
    
    \question The acceleration of a particle is described by $a(t) = 98e^{-7t}$. Find the distance equation for $x(t)$ if $v(0) = 0$ and $x(0) = -2$.

    \vfill

    \horizontalline

    \newpage

    \question Let's define a differential equation $\deriv = \dfrac{y}{x^2 + 4}$. \\

    \begin{parts}
        \part On the axis system provided, sketch the slope field for $\deriv \forcespace$ at all points plotted on the graph. You may assume that all gridlines have length 1. \\[11pt]
            \phantom{\hspace{127pt}} \begin{tikzpicture}[declare function={f(\x,\y)= 0;},scale=1.5] 
                %                       Function goes here ^^^^^ Use \x and \y.
                %                       Change scale to make bigger or smaller.
                \def\xmin{-1}       \def\xmax{1}    % Set domain and range for
                \def\ymin{-1}    \def\ymax{1} % the slopes.  
                % ymin and ymax being non-integer can help with division by zero errors.
                \def\res{1} % resolution of the slope field
                \def\size{0mm} % size of each slope in mm
                %%%%%%%%%% do not change anything below this %%%%%%%%%%
                \pgfmathsetmacro{\nx}{(\xmax-\xmin) * \res} 
                \pgfmathsetmacro{\ny}{(\ymax-\ymin) * \res} 
                \draw[help lines, color=gray!50] (\xmin -.5,\ymin -.5) grid (\xmax +.5,\ymax +.5);
                \pgfmathsetmacro{\hx}{(\xmax-\xmin)/\nx}
                \pgfmathsetmacro{\hy}{(\ymax-\ymin)/\ny}
                \foreach \i in {0,1,2,3,4,5,6,7,9,10,11,12,13,14,15,16}
                \foreach \j in {0,...,\ny}{
                        \pgfmathsetmacro{\yprime}{f({\xmin+\i*\hx},{\ymin+\j*\hy})}
                        \draw[thick, shift={({\xmin+\i*(\xmax-\xmin)/\nx},{\ymin+\j*(\ymax-\ymin)/\ny})}]
                        ($(0,0)!\size!(-.1,-.1*\yprime)$)--($(0,0)!\size!(.1,.1*\yprime)$);
                    }
                \draw[->] (\xmin-.5,0)--(\xmax+.5,0) node[below right] {\(x\)};
                \draw[->] (0,\ymin-.5)--(0,\ymax+.5) node[above left] {\(y\)};
                %%%%%%%%%%%%% and above this %%%%%%%%%%%%%%%%
                %
                % Uncomment below two lines to include a solution.
                % The function is where FUNCTION goes and is in terms of \x.
                % e.g. "(\x)^2/(\x+1)" 
                % if you want to use trig functions, wrap your argument in "deg".
                % e.g. "sin(deg(\x))%
                %
                % \clip (\xmin -.5,\ymin -.5) rectangle (\xmax +.5,\ymax +.5);
                % \draw[domain=\xmin:\xmax, smooth, variable=\x, red, ultra thick] plot ({\x}, {-cos(deg(\x)) });
                % 
                % Uncomment below line to draw a point at (POINT) e,g, (2,1)
                %
                \foreach \x in {\xmin,...,\xmax}
                \foreach \y in {\ymin,...,\ymax}{
                \filldraw[Black!70] (\x,\y) circle (1.5pt);
                }
        \end{tikzpicture}

        \vspace{11pt}

        \parthline

        \part Find the particular solution $w = f(t)$ that passes through $y(0) = -2$.

        \vfill
    \end{parts}

    \horizontalline
\end{questions}

\newpage
\stepcounter{sectioncount}

Full Name: \hfill AP Calculus BC \\[22pt]
Date: \hfill Chapter 2 Practice Test 2 \\[22pt]
\textbf{Multiple Choice Section} \hfill \textbf{20 Minutes; No Calculator} \\[11pt]

\begin{questions}
    % Question 1
    \question Which of the following statements are \textbf{true}? \begin{align*}
        & \text{I. } \int \left(x^3 + x\right)\sqrt[4]{x^4 + 2x^2 - 5} \, dx = \dfrac{1}{5}\left(x^4 - 2x^2 - 5\right)^{\frac{5}{4}} + C \\[11pt]
        & \text{II. } \int x^5\sin \left(x^6\right) \, dx = -\dfrac{1}{6}\cos \left(x^6\right) + C \\[11pt]
        & \text{III. } \int \csc (x) \, dx = \ln |\csc (x) + \cot (x)| + C
    \end{align*}

    \begin{oneparchoices}
        \choice I only 
        \choice II only
        \choice III only 
        \choice I and II only
        \choice II and III only
    \end{oneparchoices} \par \horizontalline

    %Question 2
    \question $\int \dfrac{x - 2}{x - 1} \, dx = $ \\

    \begin{oneparchoices}
        \choice $-\ln |x - 1| + C$
        \choice $x + \ln |x - 1| + C$
        \choice $x - \ln |x - 1| + C$ \\[11pt]
        \makebox[0.22\textwidth] \choice $x - \sqrt{x - 1} + C$
        \makebox[0.21\textwidth] \choice $x + \sqrt{x - 1} + C$
    \end{oneparchoices} \par \horizontalline

    %Question 3
    \question If $\deriv = \sin (x)\cos^3 (x)$ and if $y = 1$ when $x = \pi$, what is the value of $y$ when $x = 0$? \\

    \begin{oneparchoices}
        \choice $-3$
        \choice $-2$
        \choice $1$
        \choice $2$
        \choice $3$
    \end{oneparchoices} \par \horizontalline

    \newpage

    % Question 4
    \question $\int x\sqrt{1 - x^2} \, dx = $ \\
    
    \begin{oneparchoices}
        \choice $\dfrac{\left(1 - x^2\right)^{\frac{3}{2}}}{3} + C$
        \choice $-\left(1 - x^2\right)^{\frac{3}{2}} + C$
        \choice $\dfrac{x^2\left(1 - x^2\right)^{\frac{3}{2}}}{3} + C$ \\[11pt]
        \makebox[0.21\textwidth] \choice $-\dfrac{x^2\left(1 - x^2\right)^{\frac{3}{2}}}{3} + C$
        \makebox[0.20\textwidth] \choice $-\dfrac{\left(1 - x^2\right)^{\frac{3}{2}}}{3} + C$
    \end{oneparchoices} \par \horizontalline

    % Question 5
    \question Which of the following differential equations corresponds to the slope field shown in the figure below? \begin{center}
        \begin{tikzpicture}[declare function={f(\x,\y)= \x - (\y * \y);},scale=0.8] 
            %                       Function goes here ^^^^^ Use \x and \y.
            %                       Change scale to make bigger or smaller.
            \def\xmin{-4.001}       \def\xmax{4.001}    % Set domain and range for
            \def\ymin{-4.001}    \def\ymax{4.001} % the slopes.  
            % ymin and ymax being non-integer can help with division by zero errors.
            \def\res{2} % resolution of the slope field
            \def\size{2mm} % size of each slope in mm
            %%%%%%%%%% do not change anything below this %%%%%%%%%%
            \pgfmathsetmacro{\nx}{(\xmax-\xmin) * \res} 
            \pgfmathsetmacro{\ny}{(\ymax-\ymin) * \res} 
            \draw[help lines, color=gray!50] (\xmin -.5,\ymin -.5) grid (\xmax +.5,\ymax +.5);
            \pgfmathsetmacro{\hx}{(\xmax-\xmin)/\nx}
            \pgfmathsetmacro{\hy}{(\ymax-\ymin)/\ny}
            \foreach \i in {0,...,\nx}
            \foreach \j in {0,...,\ny}{
                    \pgfmathsetmacro{\yprime}{f({\xmin+\i*\hx},{\ymin+\j*\hy})}
                    \draw[thick, shift={({\xmin+\i*(\xmax-\xmin)/\nx},{\ymin+\j*(\ymax-\ymin)/\ny})}]
                    ($(0,0)!\size!(-.1,-.1*\yprime)$)--($(0,0)!\size!(.1,.1*\yprime)$);
                }
            \draw[->] (\xmin-.5,0)--(\xmax+.5,0) node[below right] {\(x\)};
            \draw[->] (0,\ymin-.5)--(0,\ymax+.5) node[above left] {\(y\)};
            %%%%%%%%%%%%% and above this %%%%%%%%%%%%%%%%
            %
            % Uncomment below two lines to include a solution.
            % The function is where FUNCTION goes and is in terms of \x.
            % e.g. "(\x)^2/(\x+1)" 
            % if you want to use trig functions, wrap your argument in "deg".
            % e.g. "sin(deg(\x))%
            %
            % \clip (\xmin -.5,\ymin -.5) rectangle (\xmax +.5,\ymax +.5);
            % \draw[domain=\xmin:\xmax, smooth, variable=\x, red, ultra thick] plot ({\x}, {-cos(deg(\x)) });
            % 
            % Uncomment below line to draw a point at (POINT) e,g, (2,1)
            %
            %\filldraw (POINT) circle (0.1);
        \end{tikzpicture} 
    \end{center} \vspace{11pt}

    \begin{oneparchoices}
        \choice $\deriv = x - y^2$
        \choice $\deriv = 1 - \dfrac{y}{x}$
        \choice $\deriv = -y^3$
        \choice $\deriv = x - \dfrac{1}{2}x^3$
        \choice $\deriv = x + y$
    \end{oneparchoices} \par \horizontalline

    %Question 6
    \question For $\int \sec^2 (x)\tan^2 (x) \, dx$, the correct u-substitution is \\

    \begin{oneparchoices}
        \choice $u = \sec (x)$ 
        \choice $u = \tan (x)$
        \choice Either (a) or (b)
        \choice Neither (a) nor (b)
    \end{oneparchoices} \par \horizontalline

    \newpage

    %Question 7
    \question Which of the following equations might be the solution to the slope field shown in the figure below? \begin{center}
        \begin{tikzpicture}[declare function={f(\x,\y)= 4 - 3 * (\x)^2;},scale=0.8] 
            %                       Function goes here ^^^^^ Use \x and \y.
            %                       Change scale to make bigger or smaller.
            \def\xmin{-4.001}       \def\xmax{4.001}    % Set domain and range for
            \def\ymin{-4.001}    \def\ymax{4.001} % the slopes.  
            % ymin and ymax being non-integer can help with division by zero errors.
            \def\res{2} % resolution of the slope field
            \def\size{2mm} % size of each slope in mm
            %%%%%%%%%% do not change anything below this %%%%%%%%%%
            \pgfmathsetmacro{\nx}{(\xmax-\xmin) * \res} 
            \pgfmathsetmacro{\ny}{(\ymax-\ymin) * \res} 
            \draw[help lines, color=gray!50] (\xmin -.5,\ymin -.5) grid (\xmax +.5,\ymax +.5);
            \pgfmathsetmacro{\hx}{(\xmax-\xmin)/\nx}
            \pgfmathsetmacro{\hy}{(\ymax-\ymin)/\ny}
            \foreach \i in {0,...,\nx}
            \foreach \j in {0,...,\ny}{
                    \pgfmathsetmacro{\yprime}{f({\xmin+\i*\hx},{\ymin+\j*\hy})}
                    \draw[thick, shift={({\xmin+\i*(\xmax-\xmin)/\nx},{\ymin+\j*(\ymax-\ymin)/\ny})}]
                    ($(0,0)!\size!(-.1,-.1*\yprime)$)--($(0,0)!\size!(.1,.1*\yprime)$);
                }
            \draw[->] (\xmin-.5,0)--(\xmax+.5,0) node[below right] {\(x\)};
            \draw[->] (0,\ymin-.5)--(0,\ymax+.5) node[above left] {\(y\)};
            %%%%%%%%%%%%% and above this %%%%%%%%%%%%%%%%
            %
            % Uncomment below two lines to include a solution.
            % The function is where FUNCTION goes and is in terms of \x.
            % e.g. "(\x)^2/(\x+1)" 
            % if you want to use trig functions, wrap your argument in "deg".
            % e.g. "sin(deg(\x))%
            %
            % \clip (\xmin -.5,\ymin -.5) rectangle (\xmax +.5,\ymax +.5);
            % \draw[domain=\xmin:\xmax, smooth, variable=\x, red, ultra thick] plot ({\x}, {-cos(deg(\x)) });
            % 
            % Uncomment below line to draw a point at (POINT) e,g, (2,1)
            %
            %\filldraw (POINT) circle (0.1);
        \end{tikzpicture} 
    \end{center} \vspace{11pt}

    \begin{oneparchoices}
        \choice $y = 4x - x^3$
        \choice $y = x^3 - 4x$
        \choice $y = 4x^4$
        \choice $y = x^3 - 15x^5$
        \choice $y = \sec (x)$
    \end{oneparchoices} \par \horizontalline

    %Question 8
    \question Identify the first mistake (if any) in this process: \begin{align*}
        & \textbf{Problem:} && \deriv = xy + x\\[11pt]
        & \text{Step 1:} && \dfrac{1}{y + 1} \, dy = x \, dx \\[5.5pt]
        & \text{Step 2:} && \ln |y + 1| = x^2 + C \\[5.5pt]
        & \text{Step 3:} && |y + 1| = e^{x^2} + C \\[5.5pt]
        & \text{Step 4:} && y = e^{x^2} + C
    \end{align*}

    \begin{oneparchoices}
        \choice Step 1
        \choice Step 2
        \choice Step 3
        \choice Step 4
        \choice No mistake
    \end{oneparchoices} \par \horizontalline
\end{questions}

\newpage
\stepcounter{sectioncount}

\textbf{Free Response Section} \hfill \textbf{35 Minutes; No Calculator} \\[11pt]

\begin{center}
    \textbf{Show All Work}
\end{center}
\vspace{11pt}

\begin{questions}
    \question Compute the following antiderivatives. \\

    \begin{parts}
        \part $\int \dfrac{t^3 - 4t - 3}{5t^{\frac{2}{3}}} \, dt$

        \vfill

        \parthline

        \part $\int \dfrac{x^2}{\left(x^3 - 1\right)^{\frac{3}{2}}} \, dx$

        \vfill

        \parthline

        \newpage

        \part $\int x\sqrt{-3x^2 + 17} \, dx$

        \vfill
    \end{parts} 

    \horizontalline

    \question The acceleration of a particle is described by $a(t) = 48t^2 - 18t + 6$. Find the distance equation for $x(t)$ if $v(1) = 1$ and $x(1) = 3$.

    \vfill

    \horizontalline

    \newpage

    \question Let's define a differential equation $\deriv = \dfrac{y - 2}{x + 1}$. \\

    \begin{parts}
        \part On the axis system provided, sketch the slope field for $\deriv \forcespace$ at all points plotted on the graph. You may assume that all gridlines have length 1. \\[11pt]
            \phantom{\hspace{92pt}} \begin{tikzpicture}[declare function={f(\x,\y)= 0;},scale=1.2] 
                %                       Function goes here ^^^^^ Use \x and \y.
                %                       Change scale to make bigger or smaller.
                \def\xmin{-2}       \def\xmax{2}    % Set domain and range for
                \def\ymin{-1}    \def\ymax{3} % the slopes.  
                % ymin and ymax being non-integer can help with division by zero errors.
                \def\res{1} % resolution of the slope field
                \def\size{0mm} % size of each slope in mm
                %%%%%%%%%% do not change anything below this %%%%%%%%%%
                \pgfmathsetmacro{\nx}{(\xmax-\xmin) * \res} 
                \pgfmathsetmacro{\ny}{(\ymax-\ymin) * \res} 
                \draw[help lines, color=gray!50] (\xmin -.5,\ymin -.5) grid (\xmax +.5,\ymax +.5);
                \pgfmathsetmacro{\hx}{(\xmax-\xmin)/\nx}
                \pgfmathsetmacro{\hy}{(\ymax-\ymin)/\ny}
                \foreach \i in {0,1,2,3,4,5,6,7,9,10,11,12,13,14,15,16}
                \foreach \j in {0,...,\ny}{
                        \pgfmathsetmacro{\yprime}{f({\xmin+\i*\hx},{\ymin+\j*\hy})}
                        \draw[thick, shift={({\xmin+\i*(\xmax-\xmin)/\nx},{\ymin+\j*(\ymax-\ymin)/\ny})}]
                        ($(0,0)!\size!(-.1,-.1*\yprime)$)--($(0,0)!\size!(.1,.1*\yprime)$);
                    }
                \draw[->] (\xmin-.5,0)--(\xmax+.5,0) node[below right] {\(x\)};
                \draw[->] (0,\ymin-.5)--(0,\ymax+.5) node[above left] {\(y\)};
                %%%%%%%%%%%%% and above this %%%%%%%%%%%%%%%%
                %
                % Uncomment below two lines to include a solution.
                % The function is where FUNCTION goes and is in terms of \x.
                % e.g. "(\x)^2/(\x+1)" 
                % if you want to use trig functions, wrap your argument in "deg".
                % e.g. "sin(deg(\x))%
                %
                % \clip (\xmin -.5,\ymin -.5) rectangle (\xmax +.5,\ymax +.5);
                % \draw[domain=\xmin:\xmax, smooth, variable=\x, red, ultra thick] plot ({\x}, {-cos(deg(\x)) });
                % 
                % Uncomment below line to draw a point at (POINT) e,g, (2,1)
                %
                \foreach \x in {\xmin,...,\xmax}
                \foreach \y in {\ymin,...,\ymax}{
                \filldraw[Black!70] (\x,\y) circle (1.5pt);
                }
        \end{tikzpicture}

        \vspace{11pt}

        \parthline

        \part If the solution curve passes through the point $(0, \, 0)$, sketch the solution curve on the same set of axes as your slope field.

        \parthline

        \part Find the equation for the solution curve of $\deriv = \dfrac{y - 2}{x + 1}$ given that $y(0) = 5$.

        \vfill
    \end{parts}

    \horizontalline
\end{questions}

\newpage

\pagecolor{white}